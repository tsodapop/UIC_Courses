
% Default to the notebook output style

    


% Inherit from the specified cell style.




    
\documentclass[11pt]{article}

    
    
    \usepackage[T1]{fontenc}
    % Nicer default font (+ math font) than Computer Modern for most use cases
    \usepackage{mathpazo}

    % Basic figure setup, for now with no caption control since it's done
    % automatically by Pandoc (which extracts ![](path) syntax from Markdown).
    \usepackage{graphicx}
    % We will generate all images so they have a width \maxwidth. This means
    % that they will get their normal width if they fit onto the page, but
    % are scaled down if they would overflow the margins.
    \makeatletter
    \def\maxwidth{\ifdim\Gin@nat@width>\linewidth\linewidth
    \else\Gin@nat@width\fi}
    \makeatother
    \let\Oldincludegraphics\includegraphics
    % Set max figure width to be 80% of text width, for now hardcoded.
    \renewcommand{\includegraphics}[1]{\Oldincludegraphics[width=.8\maxwidth]{#1}}
    % Ensure that by default, figures have no caption (until we provide a
    % proper Figure object with a Caption API and a way to capture that
    % in the conversion process - todo).
    \usepackage{caption}
    \DeclareCaptionLabelFormat{nolabel}{}
    \captionsetup{labelformat=nolabel}

    \usepackage{adjustbox} % Used to constrain images to a maximum size 
    \usepackage{xcolor} % Allow colors to be defined
    \usepackage{enumerate} % Needed for markdown enumerations to work
    \usepackage{geometry} % Used to adjust the document margins
    \usepackage{amsmath} % Equations
    \usepackage{amssymb} % Equations
    \usepackage{textcomp} % defines textquotesingle
    % Hack from http://tex.stackexchange.com/a/47451/13684:
    \AtBeginDocument{%
        \def\PYZsq{\textquotesingle}% Upright quotes in Pygmentized code
    }
    \usepackage{upquote} % Upright quotes for verbatim code
    \usepackage{eurosym} % defines \euro
    \usepackage[mathletters]{ucs} % Extended unicode (utf-8) support
    \usepackage[utf8x]{inputenc} % Allow utf-8 characters in the tex document
    \usepackage{fancyvrb} % verbatim replacement that allows latex
    \usepackage{grffile} % extends the file name processing of package graphics 
                         % to support a larger range 
    % The hyperref package gives us a pdf with properly built
    % internal navigation ('pdf bookmarks' for the table of contents,
    % internal cross-reference links, web links for URLs, etc.)
    \usepackage{hyperref}
    \usepackage{longtable} % longtable support required by pandoc >1.10
    \usepackage{booktabs}  % table support for pandoc > 1.12.2
    \usepackage[inline]{enumitem} % IRkernel/repr support (it uses the enumerate* environment)
    \usepackage[normalem]{ulem} % ulem is needed to support strikethroughs (\sout)
                                % normalem makes italics be italics, not underlines
    

    
    
    % Colors for the hyperref package
    \definecolor{urlcolor}{rgb}{0,.145,.698}
    \definecolor{linkcolor}{rgb}{.71,0.21,0.01}
    \definecolor{citecolor}{rgb}{.12,.54,.11}

    % ANSI colors
    \definecolor{ansi-black}{HTML}{3E424D}
    \definecolor{ansi-black-intense}{HTML}{282C36}
    \definecolor{ansi-red}{HTML}{E75C58}
    \definecolor{ansi-red-intense}{HTML}{B22B31}
    \definecolor{ansi-green}{HTML}{00A250}
    \definecolor{ansi-green-intense}{HTML}{007427}
    \definecolor{ansi-yellow}{HTML}{DDB62B}
    \definecolor{ansi-yellow-intense}{HTML}{B27D12}
    \definecolor{ansi-blue}{HTML}{208FFB}
    \definecolor{ansi-blue-intense}{HTML}{0065CA}
    \definecolor{ansi-magenta}{HTML}{D160C4}
    \definecolor{ansi-magenta-intense}{HTML}{A03196}
    \definecolor{ansi-cyan}{HTML}{60C6C8}
    \definecolor{ansi-cyan-intense}{HTML}{258F8F}
    \definecolor{ansi-white}{HTML}{C5C1B4}
    \definecolor{ansi-white-intense}{HTML}{A1A6B2}

    % commands and environments needed by pandoc snippets
    % extracted from the output of `pandoc -s`
    \providecommand{\tightlist}{%
      \setlength{\itemsep}{0pt}\setlength{\parskip}{0pt}}
    \DefineVerbatimEnvironment{Highlighting}{Verbatim}{commandchars=\\\{\}}
    % Add ',fontsize=\small' for more characters per line
    \newenvironment{Shaded}{}{}
    \newcommand{\KeywordTok}[1]{\textcolor[rgb]{0.00,0.44,0.13}{\textbf{{#1}}}}
    \newcommand{\DataTypeTok}[1]{\textcolor[rgb]{0.56,0.13,0.00}{{#1}}}
    \newcommand{\DecValTok}[1]{\textcolor[rgb]{0.25,0.63,0.44}{{#1}}}
    \newcommand{\BaseNTok}[1]{\textcolor[rgb]{0.25,0.63,0.44}{{#1}}}
    \newcommand{\FloatTok}[1]{\textcolor[rgb]{0.25,0.63,0.44}{{#1}}}
    \newcommand{\CharTok}[1]{\textcolor[rgb]{0.25,0.44,0.63}{{#1}}}
    \newcommand{\StringTok}[1]{\textcolor[rgb]{0.25,0.44,0.63}{{#1}}}
    \newcommand{\CommentTok}[1]{\textcolor[rgb]{0.38,0.63,0.69}{\textit{{#1}}}}
    \newcommand{\OtherTok}[1]{\textcolor[rgb]{0.00,0.44,0.13}{{#1}}}
    \newcommand{\AlertTok}[1]{\textcolor[rgb]{1.00,0.00,0.00}{\textbf{{#1}}}}
    \newcommand{\FunctionTok}[1]{\textcolor[rgb]{0.02,0.16,0.49}{{#1}}}
    \newcommand{\RegionMarkerTok}[1]{{#1}}
    \newcommand{\ErrorTok}[1]{\textcolor[rgb]{1.00,0.00,0.00}{\textbf{{#1}}}}
    \newcommand{\NormalTok}[1]{{#1}}
    
    % Additional commands for more recent versions of Pandoc
    \newcommand{\ConstantTok}[1]{\textcolor[rgb]{0.53,0.00,0.00}{{#1}}}
    \newcommand{\SpecialCharTok}[1]{\textcolor[rgb]{0.25,0.44,0.63}{{#1}}}
    \newcommand{\VerbatimStringTok}[1]{\textcolor[rgb]{0.25,0.44,0.63}{{#1}}}
    \newcommand{\SpecialStringTok}[1]{\textcolor[rgb]{0.73,0.40,0.53}{{#1}}}
    \newcommand{\ImportTok}[1]{{#1}}
    \newcommand{\DocumentationTok}[1]{\textcolor[rgb]{0.73,0.13,0.13}{\textit{{#1}}}}
    \newcommand{\AnnotationTok}[1]{\textcolor[rgb]{0.38,0.63,0.69}{\textbf{\textit{{#1}}}}}
    \newcommand{\CommentVarTok}[1]{\textcolor[rgb]{0.38,0.63,0.69}{\textbf{\textit{{#1}}}}}
    \newcommand{\VariableTok}[1]{\textcolor[rgb]{0.10,0.09,0.49}{{#1}}}
    \newcommand{\ControlFlowTok}[1]{\textcolor[rgb]{0.00,0.44,0.13}{\textbf{{#1}}}}
    \newcommand{\OperatorTok}[1]{\textcolor[rgb]{0.40,0.40,0.40}{{#1}}}
    \newcommand{\BuiltInTok}[1]{{#1}}
    \newcommand{\ExtensionTok}[1]{{#1}}
    \newcommand{\PreprocessorTok}[1]{\textcolor[rgb]{0.74,0.48,0.00}{{#1}}}
    \newcommand{\AttributeTok}[1]{\textcolor[rgb]{0.49,0.56,0.16}{{#1}}}
    \newcommand{\InformationTok}[1]{\textcolor[rgb]{0.38,0.63,0.69}{\textbf{\textit{{#1}}}}}
    \newcommand{\WarningTok}[1]{\textcolor[rgb]{0.38,0.63,0.69}{\textbf{\textit{{#1}}}}}
    
    
    % Define a nice break command that doesn't care if a line doesn't already
    % exist.
    \def\br{\hspace*{\fill} \\* }
    % Math Jax compatability definitions
    \def\gt{>}
    \def\lt{<}
    % Document parameters
    \title{lab\_01}
    
    
    

    % Pygments definitions
    
\makeatletter
\def\PY@reset{\let\PY@it=\relax \let\PY@bf=\relax%
    \let\PY@ul=\relax \let\PY@tc=\relax%
    \let\PY@bc=\relax \let\PY@ff=\relax}
\def\PY@tok#1{\csname PY@tok@#1\endcsname}
\def\PY@toks#1+{\ifx\relax#1\empty\else%
    \PY@tok{#1}\expandafter\PY@toks\fi}
\def\PY@do#1{\PY@bc{\PY@tc{\PY@ul{%
    \PY@it{\PY@bf{\PY@ff{#1}}}}}}}
\def\PY#1#2{\PY@reset\PY@toks#1+\relax+\PY@do{#2}}

\expandafter\def\csname PY@tok@w\endcsname{\def\PY@tc##1{\textcolor[rgb]{0.73,0.73,0.73}{##1}}}
\expandafter\def\csname PY@tok@c\endcsname{\let\PY@it=\textit\def\PY@tc##1{\textcolor[rgb]{0.25,0.50,0.50}{##1}}}
\expandafter\def\csname PY@tok@cp\endcsname{\def\PY@tc##1{\textcolor[rgb]{0.74,0.48,0.00}{##1}}}
\expandafter\def\csname PY@tok@k\endcsname{\let\PY@bf=\textbf\def\PY@tc##1{\textcolor[rgb]{0.00,0.50,0.00}{##1}}}
\expandafter\def\csname PY@tok@kp\endcsname{\def\PY@tc##1{\textcolor[rgb]{0.00,0.50,0.00}{##1}}}
\expandafter\def\csname PY@tok@kt\endcsname{\def\PY@tc##1{\textcolor[rgb]{0.69,0.00,0.25}{##1}}}
\expandafter\def\csname PY@tok@o\endcsname{\def\PY@tc##1{\textcolor[rgb]{0.40,0.40,0.40}{##1}}}
\expandafter\def\csname PY@tok@ow\endcsname{\let\PY@bf=\textbf\def\PY@tc##1{\textcolor[rgb]{0.67,0.13,1.00}{##1}}}
\expandafter\def\csname PY@tok@nb\endcsname{\def\PY@tc##1{\textcolor[rgb]{0.00,0.50,0.00}{##1}}}
\expandafter\def\csname PY@tok@nf\endcsname{\def\PY@tc##1{\textcolor[rgb]{0.00,0.00,1.00}{##1}}}
\expandafter\def\csname PY@tok@nc\endcsname{\let\PY@bf=\textbf\def\PY@tc##1{\textcolor[rgb]{0.00,0.00,1.00}{##1}}}
\expandafter\def\csname PY@tok@nn\endcsname{\let\PY@bf=\textbf\def\PY@tc##1{\textcolor[rgb]{0.00,0.00,1.00}{##1}}}
\expandafter\def\csname PY@tok@ne\endcsname{\let\PY@bf=\textbf\def\PY@tc##1{\textcolor[rgb]{0.82,0.25,0.23}{##1}}}
\expandafter\def\csname PY@tok@nv\endcsname{\def\PY@tc##1{\textcolor[rgb]{0.10,0.09,0.49}{##1}}}
\expandafter\def\csname PY@tok@no\endcsname{\def\PY@tc##1{\textcolor[rgb]{0.53,0.00,0.00}{##1}}}
\expandafter\def\csname PY@tok@nl\endcsname{\def\PY@tc##1{\textcolor[rgb]{0.63,0.63,0.00}{##1}}}
\expandafter\def\csname PY@tok@ni\endcsname{\let\PY@bf=\textbf\def\PY@tc##1{\textcolor[rgb]{0.60,0.60,0.60}{##1}}}
\expandafter\def\csname PY@tok@na\endcsname{\def\PY@tc##1{\textcolor[rgb]{0.49,0.56,0.16}{##1}}}
\expandafter\def\csname PY@tok@nt\endcsname{\let\PY@bf=\textbf\def\PY@tc##1{\textcolor[rgb]{0.00,0.50,0.00}{##1}}}
\expandafter\def\csname PY@tok@nd\endcsname{\def\PY@tc##1{\textcolor[rgb]{0.67,0.13,1.00}{##1}}}
\expandafter\def\csname PY@tok@s\endcsname{\def\PY@tc##1{\textcolor[rgb]{0.73,0.13,0.13}{##1}}}
\expandafter\def\csname PY@tok@sd\endcsname{\let\PY@it=\textit\def\PY@tc##1{\textcolor[rgb]{0.73,0.13,0.13}{##1}}}
\expandafter\def\csname PY@tok@si\endcsname{\let\PY@bf=\textbf\def\PY@tc##1{\textcolor[rgb]{0.73,0.40,0.53}{##1}}}
\expandafter\def\csname PY@tok@se\endcsname{\let\PY@bf=\textbf\def\PY@tc##1{\textcolor[rgb]{0.73,0.40,0.13}{##1}}}
\expandafter\def\csname PY@tok@sr\endcsname{\def\PY@tc##1{\textcolor[rgb]{0.73,0.40,0.53}{##1}}}
\expandafter\def\csname PY@tok@ss\endcsname{\def\PY@tc##1{\textcolor[rgb]{0.10,0.09,0.49}{##1}}}
\expandafter\def\csname PY@tok@sx\endcsname{\def\PY@tc##1{\textcolor[rgb]{0.00,0.50,0.00}{##1}}}
\expandafter\def\csname PY@tok@m\endcsname{\def\PY@tc##1{\textcolor[rgb]{0.40,0.40,0.40}{##1}}}
\expandafter\def\csname PY@tok@gh\endcsname{\let\PY@bf=\textbf\def\PY@tc##1{\textcolor[rgb]{0.00,0.00,0.50}{##1}}}
\expandafter\def\csname PY@tok@gu\endcsname{\let\PY@bf=\textbf\def\PY@tc##1{\textcolor[rgb]{0.50,0.00,0.50}{##1}}}
\expandafter\def\csname PY@tok@gd\endcsname{\def\PY@tc##1{\textcolor[rgb]{0.63,0.00,0.00}{##1}}}
\expandafter\def\csname PY@tok@gi\endcsname{\def\PY@tc##1{\textcolor[rgb]{0.00,0.63,0.00}{##1}}}
\expandafter\def\csname PY@tok@gr\endcsname{\def\PY@tc##1{\textcolor[rgb]{1.00,0.00,0.00}{##1}}}
\expandafter\def\csname PY@tok@ge\endcsname{\let\PY@it=\textit}
\expandafter\def\csname PY@tok@gs\endcsname{\let\PY@bf=\textbf}
\expandafter\def\csname PY@tok@gp\endcsname{\let\PY@bf=\textbf\def\PY@tc##1{\textcolor[rgb]{0.00,0.00,0.50}{##1}}}
\expandafter\def\csname PY@tok@go\endcsname{\def\PY@tc##1{\textcolor[rgb]{0.53,0.53,0.53}{##1}}}
\expandafter\def\csname PY@tok@gt\endcsname{\def\PY@tc##1{\textcolor[rgb]{0.00,0.27,0.87}{##1}}}
\expandafter\def\csname PY@tok@err\endcsname{\def\PY@bc##1{\setlength{\fboxsep}{0pt}\fcolorbox[rgb]{1.00,0.00,0.00}{1,1,1}{\strut ##1}}}
\expandafter\def\csname PY@tok@kc\endcsname{\let\PY@bf=\textbf\def\PY@tc##1{\textcolor[rgb]{0.00,0.50,0.00}{##1}}}
\expandafter\def\csname PY@tok@kd\endcsname{\let\PY@bf=\textbf\def\PY@tc##1{\textcolor[rgb]{0.00,0.50,0.00}{##1}}}
\expandafter\def\csname PY@tok@kn\endcsname{\let\PY@bf=\textbf\def\PY@tc##1{\textcolor[rgb]{0.00,0.50,0.00}{##1}}}
\expandafter\def\csname PY@tok@kr\endcsname{\let\PY@bf=\textbf\def\PY@tc##1{\textcolor[rgb]{0.00,0.50,0.00}{##1}}}
\expandafter\def\csname PY@tok@bp\endcsname{\def\PY@tc##1{\textcolor[rgb]{0.00,0.50,0.00}{##1}}}
\expandafter\def\csname PY@tok@fm\endcsname{\def\PY@tc##1{\textcolor[rgb]{0.00,0.00,1.00}{##1}}}
\expandafter\def\csname PY@tok@vc\endcsname{\def\PY@tc##1{\textcolor[rgb]{0.10,0.09,0.49}{##1}}}
\expandafter\def\csname PY@tok@vg\endcsname{\def\PY@tc##1{\textcolor[rgb]{0.10,0.09,0.49}{##1}}}
\expandafter\def\csname PY@tok@vi\endcsname{\def\PY@tc##1{\textcolor[rgb]{0.10,0.09,0.49}{##1}}}
\expandafter\def\csname PY@tok@vm\endcsname{\def\PY@tc##1{\textcolor[rgb]{0.10,0.09,0.49}{##1}}}
\expandafter\def\csname PY@tok@sa\endcsname{\def\PY@tc##1{\textcolor[rgb]{0.73,0.13,0.13}{##1}}}
\expandafter\def\csname PY@tok@sb\endcsname{\def\PY@tc##1{\textcolor[rgb]{0.73,0.13,0.13}{##1}}}
\expandafter\def\csname PY@tok@sc\endcsname{\def\PY@tc##1{\textcolor[rgb]{0.73,0.13,0.13}{##1}}}
\expandafter\def\csname PY@tok@dl\endcsname{\def\PY@tc##1{\textcolor[rgb]{0.73,0.13,0.13}{##1}}}
\expandafter\def\csname PY@tok@s2\endcsname{\def\PY@tc##1{\textcolor[rgb]{0.73,0.13,0.13}{##1}}}
\expandafter\def\csname PY@tok@sh\endcsname{\def\PY@tc##1{\textcolor[rgb]{0.73,0.13,0.13}{##1}}}
\expandafter\def\csname PY@tok@s1\endcsname{\def\PY@tc##1{\textcolor[rgb]{0.73,0.13,0.13}{##1}}}
\expandafter\def\csname PY@tok@mb\endcsname{\def\PY@tc##1{\textcolor[rgb]{0.40,0.40,0.40}{##1}}}
\expandafter\def\csname PY@tok@mf\endcsname{\def\PY@tc##1{\textcolor[rgb]{0.40,0.40,0.40}{##1}}}
\expandafter\def\csname PY@tok@mh\endcsname{\def\PY@tc##1{\textcolor[rgb]{0.40,0.40,0.40}{##1}}}
\expandafter\def\csname PY@tok@mi\endcsname{\def\PY@tc##1{\textcolor[rgb]{0.40,0.40,0.40}{##1}}}
\expandafter\def\csname PY@tok@il\endcsname{\def\PY@tc##1{\textcolor[rgb]{0.40,0.40,0.40}{##1}}}
\expandafter\def\csname PY@tok@mo\endcsname{\def\PY@tc##1{\textcolor[rgb]{0.40,0.40,0.40}{##1}}}
\expandafter\def\csname PY@tok@ch\endcsname{\let\PY@it=\textit\def\PY@tc##1{\textcolor[rgb]{0.25,0.50,0.50}{##1}}}
\expandafter\def\csname PY@tok@cm\endcsname{\let\PY@it=\textit\def\PY@tc##1{\textcolor[rgb]{0.25,0.50,0.50}{##1}}}
\expandafter\def\csname PY@tok@cpf\endcsname{\let\PY@it=\textit\def\PY@tc##1{\textcolor[rgb]{0.25,0.50,0.50}{##1}}}
\expandafter\def\csname PY@tok@c1\endcsname{\let\PY@it=\textit\def\PY@tc##1{\textcolor[rgb]{0.25,0.50,0.50}{##1}}}
\expandafter\def\csname PY@tok@cs\endcsname{\let\PY@it=\textit\def\PY@tc##1{\textcolor[rgb]{0.25,0.50,0.50}{##1}}}

\def\PYZbs{\char`\\}
\def\PYZus{\char`\_}
\def\PYZob{\char`\{}
\def\PYZcb{\char`\}}
\def\PYZca{\char`\^}
\def\PYZam{\char`\&}
\def\PYZlt{\char`\<}
\def\PYZgt{\char`\>}
\def\PYZsh{\char`\#}
\def\PYZpc{\char`\%}
\def\PYZdl{\char`\$}
\def\PYZhy{\char`\-}
\def\PYZsq{\char`\'}
\def\PYZdq{\char`\"}
\def\PYZti{\char`\~}
% for compatibility with earlier versions
\def\PYZat{@}
\def\PYZlb{[}
\def\PYZrb{]}
\makeatother


    % Exact colors from NB
    \definecolor{incolor}{rgb}{0.0, 0.0, 0.5}
    \definecolor{outcolor}{rgb}{0.545, 0.0, 0.0}



    
    % Prevent overflowing lines due to hard-to-break entities
    \sloppy 
    % Setup hyperref package
    \hypersetup{
      breaklinks=true,  % so long urls are correctly broken across lines
      colorlinks=true,
      urlcolor=urlcolor,
      linkcolor=linkcolor,
      citecolor=citecolor,
      }
    % Slightly bigger margins than the latex defaults
    
    \geometry{verbose,tmargin=1in,bmargin=1in,lmargin=1in,rmargin=1in}
    
    

    \begin{document}
    
    
    \maketitle
    
    

    
    \section{Lab Assignment 01}\label{lab-assignment-01}

The objective of this lab assignment is to review basic concepts of the
Python programming language (functions, strings, lists, dictionaries,
control flow, list comprehensions) and to introduce the main data
structures, functions, and methods of the \texttt{pandas} package for
data analysis.

\paragraph{Instructions:}\label{instructions}

Complete each task by filling in the blanks (\texttt{...}) with one or
more lines of code. Each task is worth \textbf{0.5 points} (out of
\textbf{10 points}).

\paragraph{Submission:}\label{submission}

This assignment is due \textbf{Sunday, September 22, at 11:59PM (Central
Time)}.

This assignment must be submitted on Gradescope as a \textbf{PDF file}
containing the completed code for each task and the corresponding
output. Late submissions will be accepted within \textbf{0-12} hours
after the deadline with a \textbf{0.5-point (5\%) penalty} and within
\textbf{12-24} hours after the deadline with a \textbf{2-point (20\%)
penalty}. No late submissions will be accepted more than 24 hours after
the deadline.

\textbf{This assignment is individual}. Offering or receiving any kind
of unauthorized or unacknowledged assistance is a violation of the
University's academic integrity policies, will result in a grade of zero
for the assignment, and will be subject to disciplinary action.

\paragraph{References:}\label{references}

\begin{itemize}
\tightlist
\item
  The Python Tutorial (\href{http://docs.python.org/3/tutorial/}{Link})
\item
  10 minutes to \texttt{pandas}
  (\href{http://pandas.pydata.org/pandas-docs/stable/getting_started/10min.html}{Link})
\item
  Joel Grus. \emph{Data Science from Scratch} (2019).
\end{itemize}

    \subsubsection{Part 1: Functions}\label{part-1-functions}

\textbf{Functions} in Python are defined using the keyword \texttt{def},
followed by the function name and the parenthesized list of
\textbf{parameters} or \textbf{arguments}.

The statements that form the body of the function start in the next line
and must be indented. The first statement of the function body can
optionally be a string containing the function's documentation string or
\textbf{docstring}. The use of docstrings is strongly recommended.

Most functions end with a \texttt{return} statement that returns a value
from the function. Functions without a \texttt{return} statement return
\texttt{None}.

    \begin{Verbatim}[commandchars=\\\{\}]
{\color{incolor}In [{\color{incolor}44}]:} \PY{k}{def} \PY{n+nf}{add1}\PY{p}{(}\PY{n}{x}\PY{p}{)}\PY{p}{:}
             \PY{l+s+sd}{\PYZdq{}\PYZdq{}\PYZdq{}This function adds 1 to x and returns the result.\PYZdq{}\PYZdq{}\PYZdq{}}
             \PY{k}{return} \PY{n}{x} \PY{o}{+} \PY{l+m+mi}{1}
         \PY{n}{add1}\PY{p}{(}\PY{l+m+mi}{2}\PY{p}{)} \PY{c+c1}{\PYZsh{} returns 3}
\end{Verbatim}


\begin{Verbatim}[commandchars=\\\{\}]
{\color{outcolor}Out[{\color{outcolor}44}]:} 3
\end{Verbatim}
            
    \begin{Verbatim}[commandchars=\\\{\}]
{\color{incolor}In [{\color{incolor}45}]:} \PY{n}{help}\PY{p}{(}\PY{n}{add1}\PY{p}{)} \PY{c+c1}{\PYZsh{} returns information about function add1}
\end{Verbatim}


    \begin{Verbatim}[commandchars=\\\{\}]
Help on function add1 in module \_\_main\_\_:

add1(x)
    This function adds 1 to x and returns the result.


    \end{Verbatim}

    \textbf{Task 01 (of 20): Write a function that returns the square of
\texttt{x}.}

    \begin{Verbatim}[commandchars=\\\{\}]
{\color{incolor}In [{\color{incolor}46}]:} \PY{k}{def} \PY{n+nf}{squared}\PY{p}{(}\PY{n}{x}\PY{p}{)}\PY{p}{:}
             \PY{l+s+sd}{\PYZdq{}\PYZdq{}\PYZdq{}squares the input value\PYZdq{}\PYZdq{}\PYZdq{}}
             \PY{k}{return} \PY{n}{x}\PY{o}{*}\PY{o}{*}\PY{l+m+mi}{2}
             \PY{o}{.}\PY{o}{.}\PY{o}{.}
\end{Verbatim}


    \begin{Verbatim}[commandchars=\\\{\}]
{\color{incolor}In [{\color{incolor}47}]:} \PY{n}{squared}\PY{p}{(}\PY{l+m+mi}{3}\PY{p}{)}
\end{Verbatim}


\begin{Verbatim}[commandchars=\\\{\}]
{\color{outcolor}Out[{\color{outcolor}47}]:} 9
\end{Verbatim}
            
    Short \textbf{anonymous functions} can also be defined using the keyword
\texttt{lambda}. \textbf{Lambda functions} can be used wherever a
function can be used.

    \begin{Verbatim}[commandchars=\\\{\}]
{\color{incolor}In [{\color{incolor}48}]:} \PY{n}{add2} \PY{o}{=} \PY{k}{lambda} \PY{n}{x}\PY{p}{:} \PY{n}{x} \PY{o}{+} \PY{l+m+mi}{2}
         \PY{n}{add2}\PY{p}{(}\PY{l+m+mi}{3}\PY{p}{)} \PY{c+c1}{\PYZsh{} returns 5}
\end{Verbatim}


\begin{Verbatim}[commandchars=\\\{\}]
{\color{outcolor}Out[{\color{outcolor}48}]:} 5
\end{Verbatim}
            
    \textbf{Task 02 (of 20): Write a lambda function that returns the cube
of \texttt{x}.}

    \begin{Verbatim}[commandchars=\\\{\}]
{\color{incolor}In [{\color{incolor}49}]:} \PY{n}{cubed} \PY{o}{=} \PY{k}{lambda} \PY{n}{x}\PY{p}{:} \PY{n}{x} \PY{o}{*}\PY{o}{*} \PY{l+m+mi}{3}
\end{Verbatim}


    \begin{Verbatim}[commandchars=\\\{\}]
{\color{incolor}In [{\color{incolor}50}]:} \PY{n}{cubed}\PY{p}{(}\PY{l+m+mi}{3}\PY{p}{)}
\end{Verbatim}


\begin{Verbatim}[commandchars=\\\{\}]
{\color{outcolor}Out[{\color{outcolor}50}]:} 27
\end{Verbatim}
            
    \subsubsection{Part 2: Strings}\label{part-2-strings}

\textbf{Strings} in Python can be enclosed in single quotes or double
quotes. The backslash symbol (\texttt{\textbackslash{}}) can be used to
escape quotes.

The \texttt{print()} function can be used to output a string and the
\texttt{len()} function can be used to return the \textbf{length} of a
string.

    \begin{Verbatim}[commandchars=\\\{\}]
{\color{incolor}In [{\color{incolor}51}]:} \PY{n}{string1} \PY{o}{=} \PY{l+s+s1}{\PYZsq{}}\PY{l+s+s1}{Hello}\PY{l+s+s1}{\PYZsq{}}
         \PY{n+nb}{print}\PY{p}{(}\PY{n}{string1}\PY{p}{)}
         \PY{n}{string2} \PY{o}{=} \PY{l+s+s2}{\PYZdq{}}\PY{l+s+s2}{world!}\PY{l+s+s2}{\PYZdq{}}
         \PY{n+nb}{print}\PY{p}{(}\PY{n}{string2}\PY{p}{)}
         \PY{n}{string3} \PY{o}{=} \PY{l+s+s1}{\PYZsq{}}\PY{l+s+se}{\PYZbs{}\PYZdq{}}\PY{l+s+s1}{Hello world!}\PY{l+s+se}{\PYZbs{}\PYZdq{}}\PY{l+s+s1}{\PYZsq{}}
         \PY{n+nb}{print}\PY{p}{(}\PY{n}{string3}\PY{p}{)}
\end{Verbatim}


    \begin{Verbatim}[commandchars=\\\{\}]
Hello
world!
"Hello world!"

    \end{Verbatim}

    \begin{Verbatim}[commandchars=\\\{\}]
{\color{incolor}In [{\color{incolor}52}]:} \PY{n+nb}{len}\PY{p}{(}\PY{n}{string1}\PY{p}{)} \PY{c+c1}{\PYZsh{} returns 5}
\end{Verbatim}


\begin{Verbatim}[commandchars=\\\{\}]
{\color{outcolor}Out[{\color{outcolor}52}]:} 5
\end{Verbatim}
            
    Strings can span multiple lines using three single quotes or double
quotes.

    \begin{Verbatim}[commandchars=\\\{\}]
{\color{incolor}In [{\color{incolor}53}]:} \PY{n}{string\PYZus{}multi} \PY{o}{=} \PY{l+s+s1}{\PYZsq{}\PYZsq{}\PYZsq{}}\PY{l+s+s1}{Hello}
         \PY{l+s+s1}{world!}\PY{l+s+s1}{\PYZsq{}\PYZsq{}\PYZsq{}}
         \PY{n+nb}{print}\PY{p}{(}\PY{n}{string\PYZus{}multi}\PY{p}{)}
\end{Verbatim}


    \begin{Verbatim}[commandchars=\\\{\}]
Hello
world!

    \end{Verbatim}

    Strings can be concatenated using the \texttt{+} operator and repeated
using the \texttt{*} operator.

\textbf{Task 03 (of 20): Concatenate strings \texttt{x} and \texttt{y}
and repeat string \texttt{y} two times.}

    \begin{Verbatim}[commandchars=\\\{\}]
{\color{incolor}In [{\color{incolor}54}]:} \PY{n}{x} \PY{o}{=} \PY{l+s+s2}{\PYZdq{}}\PY{l+s+s2}{good}\PY{l+s+s2}{\PYZdq{}}
         \PY{n}{y} \PY{o}{=} \PY{l+s+s2}{\PYZdq{}}\PY{l+s+s2}{bye}\PY{l+s+s2}{\PYZdq{}}
         \PY{n}{xy} \PY{o}{=} \PY{n}{x}\PY{o}{+}\PY{n}{y}
         \PY{n}{yy} \PY{o}{=} \PY{n}{y} \PY{o}{*} \PY{l+m+mi}{2}
\end{Verbatim}


    \begin{Verbatim}[commandchars=\\\{\}]
{\color{incolor}In [{\color{incolor}55}]:} \PY{n+nb}{print}\PY{p}{(}\PY{n}{xy}\PY{p}{)}
         \PY{n+nb}{print}\PY{p}{(}\PY{n}{yy}\PY{p}{)}
\end{Verbatim}


    \begin{Verbatim}[commandchars=\\\{\}]
goodbye
byebye

    \end{Verbatim}

    Strings can be \textbf{indexed}. The first character has index 0.
Negative indices start counting from the right.

Strings can also be \textbf{sliced} to obtain \textbf{substrings}. For
example, \texttt{x{[}i:j{]}} returns the substring of \texttt{x} that
starts in position \texttt{i} and ends in, \textbf{but does not
include}, position \texttt{j}. If index \texttt{i} is omitted, it
defaults to 0, and if index \texttt{j} is omitted, it defaults to the
size of the string.

\textbf{Task 04 (of 20): Return the first character, the next-to-last
character, the first three characters, and the last seven characters of
string \texttt{word}.}

    \begin{Verbatim}[commandchars=\\\{\}]
{\color{incolor}In [{\color{incolor}56}]:} \PY{n}{word} \PY{o}{=} \PY{l+s+s2}{\PYZdq{}}\PY{l+s+s2}{Introduction to Data Science}\PY{l+s+s2}{\PYZdq{}}
         \PY{n}{first} \PY{o}{=} \PY{n}{word}\PY{p}{[}\PY{l+m+mi}{0}\PY{p}{]}
         \PY{n}{next\PYZus{}to\PYZus{}last} \PY{o}{=} \PY{n}{word}\PY{p}{[}\PY{o}{\PYZhy{}}\PY{l+m+mi}{2}\PY{p}{]}
         \PY{n}{first\PYZus{}three} \PY{o}{=} \PY{n}{word}\PY{p}{[}\PY{p}{:}\PY{l+m+mi}{3}\PY{p}{]}
         \PY{n}{last\PYZus{}seven} \PY{o}{=} \PY{n}{word}\PY{p}{[}\PY{o}{\PYZhy{}}\PY{l+m+mi}{7}\PY{p}{:}\PY{p}{]}
\end{Verbatim}


    \begin{Verbatim}[commandchars=\\\{\}]
{\color{incolor}In [{\color{incolor}57}]:} \PY{n+nb}{print}\PY{p}{(}\PY{n}{first}\PY{p}{)}
         \PY{n+nb}{print}\PY{p}{(}\PY{n}{next\PYZus{}to\PYZus{}last}\PY{p}{)}
         \PY{n+nb}{print}\PY{p}{(}\PY{n}{first\PYZus{}three}\PY{p}{)}
         \PY{n+nb}{print}\PY{p}{(}\PY{n}{last\PYZus{}seven}\PY{p}{)}
\end{Verbatim}


    \begin{Verbatim}[commandchars=\\\{\}]
I
c
Int
Science

    \end{Verbatim}

    Python strings are \textbf{immutable}; that is, they cannot be changed.
Trying to assign a value to a position in a string results in an error.

    \begin{Verbatim}[commandchars=\\\{\}]
{\color{incolor}In [{\color{incolor} }]:} \PY{n}{word}\PY{p}{[}\PY{l+m+mi}{0}\PY{p}{]} \PY{o}{=} \PY{l+s+s1}{\PYZsq{}}\PY{l+s+s1}{i}\PY{l+s+s1}{\PYZsq{}} \PY{c+c1}{\PYZsh{} results in an error}
\end{Verbatim}


    \subsubsection{Part 3: Lists}\label{part-3-lists}

\textbf{Lists} are one the most useful data structures in Python. Lists
can be written as a comma-separated list of \textbf{items} between
brackets.

The \texttt{print()} function can be used to output a list, the
\texttt{len()} function can be used to return the \textbf{number of
items} in a list, and the \texttt{in} operator can be used to check
whether an item is in a list.

    \begin{Verbatim}[commandchars=\\\{\}]
{\color{incolor}In [{\color{incolor}21}]:} \PY{n}{even\PYZus{}list} \PY{o}{=} \PY{p}{[}\PY{l+m+mi}{0}\PY{p}{,} \PY{l+m+mi}{2}\PY{p}{,} \PY{l+m+mi}{4}\PY{p}{,} \PY{l+m+mi}{6}\PY{p}{,} \PY{l+m+mi}{8}\PY{p}{,} \PY{l+m+mi}{10}\PY{p}{]}
         \PY{n+nb}{print}\PY{p}{(}\PY{n}{even\PYZus{}list}\PY{p}{)}
\end{Verbatim}


    \begin{Verbatim}[commandchars=\\\{\}]
[0, 2, 4, 6, 8, 10]

    \end{Verbatim}

    \begin{Verbatim}[commandchars=\\\{\}]
{\color{incolor}In [{\color{incolor}22}]:} \PY{n+nb}{len}\PY{p}{(}\PY{n}{even\PYZus{}list}\PY{p}{)} \PY{c+c1}{\PYZsh{} returns 6}
\end{Verbatim}


\begin{Verbatim}[commandchars=\\\{\}]
{\color{outcolor}Out[{\color{outcolor}22}]:} 6
\end{Verbatim}
            
    \begin{Verbatim}[commandchars=\\\{\}]
{\color{incolor}In [{\color{incolor}23}]:} \PY{l+m+mi}{1} \PY{o+ow}{in} \PY{n}{even\PYZus{}list} \PY{c+c1}{\PYZsh{} returns False}
\end{Verbatim}


\begin{Verbatim}[commandchars=\\\{\}]
{\color{outcolor}Out[{\color{outcolor}23}]:} False
\end{Verbatim}
            
    Like strings, lists can be \textbf{indexed} and \textbf{sliced}.

\textbf{Task 05 (of 20): Return the second item, the last item, the
middle two items, and the items in even positions of list
\texttt{even\_list}.} \emph{Hint:} A slice can take a third parameter
that specifies its \textbf{stride}.

    \begin{Verbatim}[commandchars=\\\{\}]
{\color{incolor}In [{\color{incolor}36}]:} \PY{n}{second} \PY{o}{=} \PY{n}{even\PYZus{}list}\PY{p}{[}\PY{l+m+mi}{1}\PY{p}{]}
         \PY{n}{last} \PY{o}{=} \PY{n}{even\PYZus{}list}\PY{p}{[}\PY{o}{\PYZhy{}}\PY{l+m+mi}{1}\PY{p}{]}
         \PY{n}{middle\PYZus{}two} \PY{o}{=} \PY{n}{even\PYZus{}list}\PY{p}{[}\PY{n+nb}{int}\PY{p}{(}\PY{n+nb}{len}\PY{p}{(}\PY{n}{even\PYZus{}list}\PY{p}{)}\PY{o}{/}\PY{l+m+mi}{2}\PY{p}{)}\PY{o}{\PYZhy{}}\PY{l+m+mi}{1}\PY{p}{:}\PY{n+nb}{int}\PY{p}{(}\PY{n+nb}{len}\PY{p}{(}\PY{n}{even\PYZus{}list}\PY{p}{)}\PY{o}{/}\PY{l+m+mi}{2}\PY{p}{)}\PY{o}{+}\PY{l+m+mi}{1}\PY{p}{]}
         \PY{n}{even\PYZus{}positions} \PY{o}{=} \PY{n}{even\PYZus{}list}\PY{p}{[}\PY{p}{:}\PY{p}{:}\PY{l+m+mi}{2}\PY{p}{]}
\end{Verbatim}


    \begin{Verbatim}[commandchars=\\\{\}]
{\color{incolor}In [{\color{incolor}37}]:} \PY{n+nb}{print}\PY{p}{(}\PY{n}{second}\PY{p}{)}
         \PY{n+nb}{print}\PY{p}{(}\PY{n}{last}\PY{p}{)}
         \PY{n+nb}{print}\PY{p}{(}\PY{n}{middle\PYZus{}two}\PY{p}{)}
         \PY{n+nb}{print}\PY{p}{(}\PY{n}{even\PYZus{}positions}\PY{p}{)}
\end{Verbatim}


    \begin{Verbatim}[commandchars=\\\{\}]
2
10
[4, 6]
[0, 4, 8]

    \end{Verbatim}

    Unlike strings, lists are \textbf{mutable}; that is, their content can
be changed. It is also possible to add a new item at the end of a list
using the \texttt{append()} method.

    \begin{Verbatim}[commandchars=\\\{\}]
{\color{incolor}In [{\color{incolor}38}]:} \PY{n}{even\PYZus{}list}\PY{o}{.}\PY{n}{append}\PY{p}{(}\PY{l+m+mi}{12}\PY{p}{)} \PY{c+c1}{\PYZsh{} appends 12 to end of list}
         \PY{n}{even\PYZus{}list}\PY{o}{.}\PY{n}{append}\PY{p}{(}\PY{l+m+mi}{15}\PY{p}{)} \PY{c+c1}{\PYZsh{} appends 14 to end of list}
         \PY{n+nb}{print}\PY{p}{(}\PY{n}{even\PYZus{}list}\PY{p}{)}
         \PY{n}{even\PYZus{}list}\PY{p}{[}\PY{o}{\PYZhy{}}\PY{l+m+mi}{1}\PY{p}{]} \PY{o}{=} \PY{l+m+mi}{14} \PY{c+c1}{\PYZsh{} changes last element of list}
         \PY{n+nb}{print}\PY{p}{(}\PY{n}{even\PYZus{}list}\PY{p}{)}
         \PY{n}{even\PYZus{}list}\PY{p}{[}\PY{o}{\PYZhy{}}\PY{l+m+mi}{2}\PY{p}{:}\PY{p}{]} \PY{o}{=} \PY{p}{[}\PY{p}{]} \PY{c+c1}{\PYZsh{} removes last two elements of list}
         \PY{n+nb}{print}\PY{p}{(}\PY{n}{even\PYZus{}list}\PY{p}{)}
\end{Verbatim}


    \begin{Verbatim}[commandchars=\\\{\}]
[0, 2, 4, 6, 8, 10, 12, 15]
[0, 2, 4, 6, 8, 10, 12, 14]
[0, 2, 4, 6, 8, 10]

    \end{Verbatim}

    Lists can be \textbf{sorted} using the \texttt{sort} method (in-place)
or the \texttt{sorted()} function (not-in-place)

    \begin{Verbatim}[commandchars=\\\{\}]
{\color{incolor}In [{\color{incolor}58}]:} \PY{n}{some\PYZus{}list} \PY{o}{=} \PY{p}{[}\PY{l+m+mi}{2}\PY{p}{,} \PY{o}{\PYZhy{}}\PY{l+m+mi}{5}\PY{p}{,} \PY{l+m+mi}{11}\PY{p}{,} \PY{l+m+mi}{8}\PY{p}{,} \PY{o}{\PYZhy{}}\PY{l+m+mi}{3}\PY{p}{]}
         \PY{n}{some\PYZus{}list\PYZus{}sorted} \PY{o}{=} \PY{n+nb}{sorted}\PY{p}{(}\PY{n}{some\PYZus{}list}\PY{p}{)} \PY{c+c1}{\PYZsh{} sort items from smallest to largest}
         \PY{n+nb}{print}\PY{p}{(}\PY{n}{some\PYZus{}list\PYZus{}sorted}\PY{p}{)}
\end{Verbatim}


    \begin{Verbatim}[commandchars=\\\{\}]
[-5, -3, 2, 8, 11]

    \end{Verbatim}

    \textbf{Task 06 (of 20): Sort the items of list \texttt{some\_list} by
absolute value from largest to smallest.} \emph{Hint:} Check the
parameters of the \texttt{sorted()} function.

    \begin{Verbatim}[commandchars=\\\{\}]
{\color{incolor}In [{\color{incolor}59}]:} \PY{n}{some\PYZus{}list\PYZus{}sorted\PYZus{}again} \PY{o}{=} \PY{n+nb}{sorted}\PY{p}{(}\PY{n}{some\PYZus{}list}\PY{p}{,} \PY{n}{key}\PY{o}{=}\PY{n+nb}{abs}\PY{p}{)}
\end{Verbatim}


    \begin{Verbatim}[commandchars=\\\{\}]
{\color{incolor}In [{\color{incolor}60}]:} \PY{n+nb}{print}\PY{p}{(}\PY{n}{some\PYZus{}list\PYZus{}sorted\PYZus{}again}\PY{p}{)}
\end{Verbatim}


    \begin{Verbatim}[commandchars=\\\{\}]
[2, -3, -5, 8, 11]

    \end{Verbatim}

    \subsubsection{Part 4: Dictionaries}\label{part-4-dictionaries}

Another useful data structure in Python are \textbf{dictionaries}, which
are sets of \textbf{keys} associated with \textbf{values}. Keys must be
unique and can be of any immutable type, such as strings and numbers.
Dictionaries can be written as a comma-separated list of
\texttt{key:\ value} pairs between braces.

The \texttt{print()} function can be used to output a dictionary, the
\texttt{len()} function can be used to return the \textbf{number of
key-value pairs} in a dictionary, the \texttt{list()} function can be
used to return a list of all keys in a dictionary, and the \texttt{in}
operator can be used to check whether a key is in a dictionary.

    \begin{Verbatim}[commandchars=\\\{\}]
{\color{incolor}In [{\color{incolor}61}]:} \PY{n}{grades} \PY{o}{=} \PY{p}{\PYZob{}}\PY{l+s+s1}{\PYZsq{}}\PY{l+s+s1}{John}\PY{l+s+s1}{\PYZsq{}}\PY{p}{:} \PY{l+m+mi}{85}\PY{p}{,} \PY{l+s+s1}{\PYZsq{}}\PY{l+s+s1}{Ana}\PY{l+s+s1}{\PYZsq{}}\PY{p}{:} \PY{l+m+mi}{97}\PY{p}{,} \PY{l+s+s1}{\PYZsq{}}\PY{l+s+s1}{Rob}\PY{l+s+s1}{\PYZsq{}}\PY{p}{:} \PY{l+m+mi}{78}\PY{p}{\PYZcb{}}
         \PY{n+nb}{print}\PY{p}{(}\PY{n}{grades}\PY{p}{)}
\end{Verbatim}


    \begin{Verbatim}[commandchars=\\\{\}]
\{'John': 85, 'Ana': 97, 'Rob': 78\}

    \end{Verbatim}

    \begin{Verbatim}[commandchars=\\\{\}]
{\color{incolor}In [{\color{incolor}62}]:} \PY{n+nb}{len}\PY{p}{(}\PY{n}{grades}\PY{p}{)} \PY{c+c1}{\PYZsh{} returns 3}
\end{Verbatim}


\begin{Verbatim}[commandchars=\\\{\}]
{\color{outcolor}Out[{\color{outcolor}62}]:} 3
\end{Verbatim}
            
    \begin{Verbatim}[commandchars=\\\{\}]
{\color{incolor}In [{\color{incolor}63}]:} \PY{n+nb}{list}\PY{p}{(}\PY{n}{grades}\PY{p}{)} \PY{c+c1}{\PYZsh{} Returns \PYZsq{}John\PYZsq{}, \PYZsq{}Ana\PYZsq{}, and \PYZsq{}Rob\PYZsq{}}
\end{Verbatim}


\begin{Verbatim}[commandchars=\\\{\}]
{\color{outcolor}Out[{\color{outcolor}63}]:} ['John', 'Ana', 'Rob']
\end{Verbatim}
            
    \begin{Verbatim}[commandchars=\\\{\}]
{\color{incolor}In [{\color{incolor}64}]:} \PY{l+s+s1}{\PYZsq{}}\PY{l+s+s1}{Sue}\PY{l+s+s1}{\PYZsq{}} \PY{o+ow}{in} \PY{n}{grades} \PY{c+c1}{\PYZsh{} returns False}
\end{Verbatim}


\begin{Verbatim}[commandchars=\\\{\}]
{\color{outcolor}Out[{\color{outcolor}64}]:} False
\end{Verbatim}
            
    Trying to access a key that is not in a dictionary results in an error.

    \begin{Verbatim}[commandchars=\\\{\}]
{\color{incolor}In [{\color{incolor}65}]:} \PY{n+nb}{print}\PY{p}{(}\PY{n}{grades}\PY{p}{[}\PY{l+s+s1}{\PYZsq{}}\PY{l+s+s1}{Sue}\PY{l+s+s1}{\PYZsq{}}\PY{p}{]}\PY{p}{)} \PY{c+c1}{\PYZsh{} results in an error}
\end{Verbatim}


    \begin{Verbatim}[commandchars=\\\{\}]

        ---------------------------------------------------------------------------

        KeyError                                  Traceback (most recent call last)

        <ipython-input-65-7468b91773fd> in <module>
    ----> 1 print(grades['Sue']) \# results in an error
    

        KeyError: 'Sue'

    \end{Verbatim}

    \textbf{Task 07 (of 20): Change Rob's grade to 88 and add Sue to
dictionary \texttt{grades}. Sue's grade is 90.}

    \begin{Verbatim}[commandchars=\\\{\}]
{\color{incolor}In [{\color{incolor}66}]:} \PY{n}{grades}\PY{p}{[}\PY{l+s+s1}{\PYZsq{}}\PY{l+s+s1}{Rob}\PY{l+s+s1}{\PYZsq{}}\PY{p}{]} \PY{o}{=} \PY{l+m+mi}{88}
         \PY{n}{grades}\PY{p}{[}\PY{l+s+s1}{\PYZsq{}}\PY{l+s+s1}{Sue}\PY{l+s+s1}{\PYZsq{}}\PY{p}{]} \PY{o}{=} \PY{l+m+mi}{90}
\end{Verbatim}


    \begin{Verbatim}[commandchars=\\\{\}]
{\color{incolor}In [{\color{incolor}67}]:} \PY{n+nb}{print}\PY{p}{(}\PY{n}{grades}\PY{p}{)}
\end{Verbatim}


    \begin{Verbatim}[commandchars=\\\{\}]
\{'John': 85, 'Ana': 97, 'Rob': 88, 'Sue': 90\}

    \end{Verbatim}

    \textbf{Task 08 (of 20): Delete John from dictionary \texttt{grades}
using the \texttt{del} statement.}

    \begin{Verbatim}[commandchars=\\\{\}]
{\color{incolor}In [{\color{incolor}69}]:} \PY{k}{del} \PY{n}{grades}\PY{p}{[}\PY{l+s+s1}{\PYZsq{}}\PY{l+s+s1}{John}\PY{l+s+s1}{\PYZsq{}}\PY{p}{]}
\end{Verbatim}


    \begin{Verbatim}[commandchars=\\\{\}]
{\color{incolor}In [{\color{incolor}70}]:} \PY{n+nb}{print}\PY{p}{(}\PY{n}{grades}\PY{p}{)}
\end{Verbatim}


    \begin{Verbatim}[commandchars=\\\{\}]
\{'Ana': 97, 'Rob': 88, 'Sue': 90\}

    \end{Verbatim}

    \subsubsection{Part 5: Control Flow}\label{part-5-control-flow}

As in other programming languages, we can write \textbf{\texttt{if}},
\textbf{\texttt{while}}, and \textbf{\texttt{for}} statements in Python.

An \texttt{if} statement can be written using the keywords \texttt{if},
\texttt{elif} (short for \texttt{else\ if}), and \texttt{else}.

    \begin{Verbatim}[commandchars=\\\{\}]
{\color{incolor}In [{\color{incolor}71}]:} \PY{n}{x} \PY{o}{=} \PY{l+m+mi}{1}
         \PY{k}{if} \PY{n}{x} \PY{o}{\PYZgt{}} \PY{l+m+mi}{0}\PY{p}{:}
             \PY{n+nb}{print}\PY{p}{(}\PY{l+s+s2}{\PYZdq{}}\PY{l+s+s2}{Positive}\PY{l+s+s2}{\PYZdq{}}\PY{p}{)}
         \PY{k}{elif} \PY{n}{x} \PY{o}{\PYZlt{}} \PY{l+m+mi}{0}\PY{p}{:}
             \PY{n+nb}{print}\PY{p}{(}\PY{l+s+s2}{\PYZdq{}}\PY{l+s+s2}{Negative}\PY{l+s+s2}{\PYZdq{}}\PY{p}{)}
         \PY{k}{else}\PY{p}{:}
             \PY{n+nb}{print}\PY{p}{(}\PY{l+s+s2}{\PYZdq{}}\PY{l+s+s2}{Zero}\PY{l+s+s2}{\PYZdq{}}\PY{p}{)}
\end{Verbatim}


    \begin{Verbatim}[commandchars=\\\{\}]
Positive

    \end{Verbatim}

    \textbf{Task 09 (of 20): Write a function, using an \texttt{if}
statement, that returns \texttt{True} if \texttt{x} is even and
\texttt{False} if \texttt{x} is odd.}

    \begin{Verbatim}[commandchars=\\\{\}]
{\color{incolor}In [{\color{incolor}73}]:} \PY{k}{def} \PY{n+nf}{is\PYZus{}even}\PY{p}{(}\PY{n}{x}\PY{p}{)}\PY{p}{:}
             \PY{k}{if} \PY{n}{x} \PY{o}{\PYZpc{}} \PY{l+m+mi}{2} \PY{o}{==} \PY{l+m+mi}{0}\PY{p}{:}
                 \PY{k}{return} \PY{k+kc}{True}
             \PY{k}{elif} \PY{n}{x} \PY{o}{\PYZpc{}} \PY{l+m+mi}{2} \PY{o}{==} \PY{l+m+mi}{1}\PY{p}{:}
                 \PY{k}{return} \PY{k+kc}{False}
             \PY{k}{else}\PY{p}{:} \PY{c+c1}{\PYZsh{}no specification on if it is not odd or even, added a final else statement with None }
                 \PY{k}{return} \PY{k+kc}{None}
\end{Verbatim}


    \begin{Verbatim}[commandchars=\\\{\}]
{\color{incolor}In [{\color{incolor}74}]:} \PY{n+nb}{print}\PY{p}{(}\PY{n}{is\PYZus{}even}\PY{p}{(}\PY{l+m+mi}{2}\PY{p}{)}\PY{p}{)}
         \PY{n+nb}{print}\PY{p}{(}\PY{n}{is\PYZus{}even}\PY{p}{(}\PY{l+m+mi}{5}\PY{p}{)}\PY{p}{)}
\end{Verbatim}


    \begin{Verbatim}[commandchars=\\\{\}]
True
False

    \end{Verbatim}

    A \texttt{while} statement executes as long as a condition is
\texttt{True}.

    \begin{Verbatim}[commandchars=\\\{\}]
{\color{incolor}In [{\color{incolor}75}]:} \PY{n}{x} \PY{o}{=} \PY{l+m+mi}{1}
         \PY{k}{while} \PY{n}{x} \PY{o}{\PYZlt{}} \PY{l+m+mi}{5}\PY{p}{:}
             \PY{n+nb}{print}\PY{p}{(}\PY{n}{x}\PY{p}{)}
             \PY{n}{x} \PY{o}{=} \PY{n}{x} \PY{o}{+} \PY{l+m+mi}{1}
\end{Verbatim}


    \begin{Verbatim}[commandchars=\\\{\}]
1
2
3
4

    \end{Verbatim}

    \textbf{Task 10 (of 20): Write a \texttt{while} statement that prints
and then squares \texttt{x} as long as \texttt{x} is less than 100.}

    \begin{Verbatim}[commandchars=\\\{\}]
{\color{incolor}In [{\color{incolor}76}]:} \PY{n}{x} \PY{o}{=} \PY{l+m+mi}{2}
         \PY{k}{while} \PY{n}{x} \PY{o}{\PYZlt{}} \PY{l+m+mi}{100}\PY{p}{:}
             \PY{n+nb}{print}\PY{p}{(}\PY{n}{x}\PY{p}{)}
             \PY{n}{x} \PY{o}{=} \PY{n}{x} \PY{o}{*}\PY{o}{*} \PY{l+m+mi}{2}
\end{Verbatim}


    \begin{Verbatim}[commandchars=\\\{\}]
2
4
16

    \end{Verbatim}

    A \texttt{for} statement iterates over the items of a sequence, such as
a list or a string, in the order that they appear in the sequence.

    \begin{Verbatim}[commandchars=\\\{\}]
{\color{incolor}In [{\color{incolor}77}]:} \PY{n}{words} \PY{o}{=} \PY{p}{[}\PY{l+s+s1}{\PYZsq{}}\PY{l+s+s1}{introduction}\PY{l+s+s1}{\PYZsq{}}\PY{p}{,} \PY{l+s+s1}{\PYZsq{}}\PY{l+s+s1}{to}\PY{l+s+s1}{\PYZsq{}}\PY{p}{,} \PY{l+s+s1}{\PYZsq{}}\PY{l+s+s1}{data}\PY{l+s+s1}{\PYZsq{}}\PY{p}{,} \PY{l+s+s1}{\PYZsq{}}\PY{l+s+s1}{science}\PY{l+s+s1}{\PYZsq{}}\PY{p}{]}
         \PY{k}{for} \PY{n}{w} \PY{o+ow}{in} \PY{n}{words}\PY{p}{:}
             \PY{n+nb}{print}\PY{p}{(}\PY{n}{w}\PY{p}{,} \PY{n+nb}{len}\PY{p}{(}\PY{n}{w}\PY{p}{)}\PY{p}{)}
\end{Verbatim}


    \begin{Verbatim}[commandchars=\\\{\}]
introduction 12
to 2
data 4
science 7

    \end{Verbatim}

    \textbf{Task 11 (of 20): Write a \texttt{for} statement that iterates
over the characters in string \texttt{long\_word} and prints those that
are vowels.}

    \begin{Verbatim}[commandchars=\\\{\}]
{\color{incolor}In [{\color{incolor}78}]:} \PY{n}{long\PYZus{}word} \PY{o}{=} \PY{l+s+s2}{\PYZdq{}}\PY{l+s+s2}{computation}\PY{l+s+s2}{\PYZdq{}}
         \PY{k}{for} \PY{n}{character} \PY{o+ow}{in} \PY{n}{long\PYZus{}word}\PY{p}{:}
             \PY{k}{if} \PY{n}{character} \PY{o+ow}{in} \PY{p}{[}\PY{l+s+s1}{\PYZsq{}}\PY{l+s+s1}{a}\PY{l+s+s1}{\PYZsq{}}\PY{p}{,}\PY{l+s+s1}{\PYZsq{}}\PY{l+s+s1}{e}\PY{l+s+s1}{\PYZsq{}}\PY{p}{,}\PY{l+s+s1}{\PYZsq{}}\PY{l+s+s1}{i}\PY{l+s+s1}{\PYZsq{}}\PY{p}{,}\PY{l+s+s1}{\PYZsq{}}\PY{l+s+s1}{o}\PY{l+s+s1}{\PYZsq{}}\PY{p}{,}\PY{l+s+s1}{\PYZsq{}}\PY{l+s+s1}{u}\PY{l+s+s1}{\PYZsq{}}\PY{p}{]}\PY{p}{:}
                 \PY{n+nb}{print}\PY{p}{(}\PY{n}{character}\PY{p}{)}
\end{Verbatim}


    \begin{Verbatim}[commandchars=\\\{\}]
o
u
a
i
o

    \end{Verbatim}

    A \texttt{for} statement can also be used to iterate over the key-value
pairs in a dictionary.

    \begin{Verbatim}[commandchars=\\\{\}]
{\color{incolor}In [{\color{incolor}79}]:} \PY{k}{for} \PY{n}{student}\PY{p}{,} \PY{n}{grade} \PY{o+ow}{in} \PY{n}{grades}\PY{o}{.}\PY{n}{items}\PY{p}{(}\PY{p}{)}\PY{p}{:}
             \PY{n+nb}{print}\PY{p}{(}\PY{l+s+s2}{\PYZdq{}}\PY{l+s+s2}{The grade of}\PY{l+s+s2}{\PYZdq{}}\PY{p}{,} \PY{n}{student}\PY{p}{,} \PY{l+s+s2}{\PYZdq{}}\PY{l+s+s2}{is}\PY{l+s+s2}{\PYZdq{}}\PY{p}{,} \PY{n}{grade}\PY{p}{)}
\end{Verbatim}


    \begin{Verbatim}[commandchars=\\\{\}]
The grade of Ana is 97
The grade of Rob is 88
The grade of Sue is 90

    \end{Verbatim}

    To iterate over a sequence of numbers, the \texttt{range()} function can
be used. For example, \texttt{range(10)} returns a sequence from 0 to 9
and \texttt{range(5,\ 10)} returns a sequence from 5 to 9.

\textbf{Task 12 (of 20): Write a \texttt{for} statement, using the
\texttt{range()} function, that iterates over the first 10 positive
integers and prints those that are multiples of 3.}

    \begin{Verbatim}[commandchars=\\\{\}]
{\color{incolor}In [{\color{incolor}81}]:} \PY{k}{for} \PY{n}{i} \PY{o+ow}{in} \PY{n+nb}{range}\PY{p}{(}\PY{l+m+mi}{1}\PY{p}{,}\PY{l+m+mi}{11}\PY{p}{)}\PY{p}{:} \PY{c+c1}{\PYZsh{}under assumption that positive integers begin at 1, and we go from 1\PYZhy{}10, meaning end is 11}
             \PY{k}{if} \PY{n}{i} \PY{o}{\PYZpc{}} \PY{l+m+mi}{3} \PY{o}{==} \PY{l+m+mi}{0}\PY{p}{:}
                 \PY{n+nb}{print}\PY{p}{(}\PY{n}{i}\PY{p}{)}
\end{Verbatim}


    \begin{Verbatim}[commandchars=\\\{\}]
3
6
9

    \end{Verbatim}

    \subsubsection{Part 6: List
Comprehensions}\label{part-6-list-comprehensions}

\textbf{List comprehensions} provide a concise way to create a list
where each item satisfies a certain condition and/or is the result of an
operation applied to the items of another list.

A list comprehension is written between brackets and contains an
expression and one or more \texttt{for} statements followed by zero or
more \texttt{if} statements.

    \begin{Verbatim}[commandchars=\\\{\}]
{\color{incolor}In [{\color{incolor}82}]:} \PY{n}{odd\PYZus{}list} \PY{o}{=} \PY{p}{[}\PY{n}{x} \PY{k}{for} \PY{n}{x} \PY{o+ow}{in} \PY{n+nb}{range}\PY{p}{(}\PY{l+m+mi}{10}\PY{p}{)} \PY{k}{if} \PY{n}{x} \PY{o}{\PYZpc{}} \PY{l+m+mi}{2} \PY{o}{!=} \PY{l+m+mi}{0}\PY{p}{]}
         \PY{n+nb}{print}\PY{p}{(}\PY{n}{odd\PYZus{}list}\PY{p}{)}
\end{Verbatim}


    \begin{Verbatim}[commandchars=\\\{\}]
[1, 3, 5, 7, 9]

    \end{Verbatim}

    \textbf{Task 13 (of 20): Write a list comprehension that creates a list
containing the squares of the items in list \texttt{odd\_list}.}

    \begin{Verbatim}[commandchars=\\\{\}]
{\color{incolor}In [{\color{incolor}84}]:} \PY{n}{odd\PYZus{}squared\PYZus{}list} \PY{o}{=} \PY{p}{[}\PY{n}{val} \PY{o}{*}\PY{o}{*} \PY{l+m+mi}{2} \PY{k}{for} \PY{n}{val} \PY{o+ow}{in} \PY{n}{odd\PYZus{}list}\PY{p}{]}
         \PY{n+nb}{print}\PY{p}{(}\PY{n}{odd\PYZus{}squared\PYZus{}list}\PY{p}{)}
\end{Verbatim}


    \begin{Verbatim}[commandchars=\\\{\}]
[1, 9, 25, 49, 81]

    \end{Verbatim}

    \textbf{Task 14 (of 20): Write a list comprehension that creates a list
containing all pairs of integers \texttt{(x,\ y)} where 0 \(\leq\)
\texttt{x} \(\leq\) 3 and \texttt{x} \(\leq\) \texttt{y} \(\leq\) 3. For
example, \texttt{(0,\ 0)} and \texttt{(1,\ 3)} should be in the list.}
\emph{Hint:} Use two \texttt{for} statements and the \texttt{range()}
function.

    \begin{Verbatim}[commandchars=\\\{\}]
{\color{incolor}In [{\color{incolor}86}]:} \PY{n}{pairs\PYZus{}list} \PY{o}{=} \PY{p}{[}\PY{p}{(}\PY{n}{n}\PY{p}{,}\PY{n}{m}\PY{p}{)} \PY{k}{for} \PY{n}{n} \PY{o+ow}{in} \PY{n+nb}{range}\PY{p}{(}\PY{l+m+mi}{0}\PY{p}{,}\PY{l+m+mi}{4}\PY{p}{)} \PY{k}{for} \PY{n}{m} \PY{o+ow}{in} \PY{n+nb}{range}\PY{p}{(}\PY{l+m+mi}{0}\PY{p}{,}\PY{l+m+mi}{4}\PY{p}{)}\PY{p}{]}
         \PY{n+nb}{print}\PY{p}{(}\PY{n}{pairs\PYZus{}list}\PY{p}{)}
\end{Verbatim}


    \begin{Verbatim}[commandchars=\\\{\}]
[(0, 0), (0, 1), (0, 2), (0, 3), (1, 0), (1, 1), (1, 2), (1, 3), (2, 0), (2, 1), (2, 2), (2, 3), (3, 0), (3, 1), (3, 2), (3, 3)]

    \end{Verbatim}

    \subsubsection{Part 7: pandas - Data
Structures}\label{part-7-pandas---data-structures}

\textbf{\texttt{pandas}} is a Python package for \textbf{data analysis}.
It is well suited for analyzing tabular data, such as SQL tables or
Excel spreadsheets, and it provides functions and methods for easily
manipulating (reshaping, slicing, merging, etc.) datasets.

pandas has two primary data structures: \textbf{\texttt{Series}} and
\textbf{\texttt{DataFrames}}. A \texttt{Series} is a one-dimensional
homogeneously-typed array and a \texttt{DataFrame} is a two-dimensional
potentially heterogeneously-typed table.

    \begin{Verbatim}[commandchars=\\\{\}]
{\color{incolor}In [{\color{incolor}1}]:} \PY{k+kn}{import} \PY{n+nn}{numpy} \PY{k}{as} \PY{n+nn}{np}
        \PY{k+kn}{import} \PY{n+nn}{pandas} \PY{k}{as} \PY{n+nn}{pd}
\end{Verbatim}


    A \texttt{Series} can be created by passing a list of values.

    \begin{Verbatim}[commandchars=\\\{\}]
{\color{incolor}In [{\color{incolor}2}]:} \PY{n}{s} \PY{o}{=} \PY{n}{pd}\PY{o}{.}\PY{n}{Series}\PY{p}{(}\PY{p}{[}\PY{l+m+mi}{0}\PY{p}{,} \PY{l+m+mi}{2}\PY{p}{,} \PY{l+m+mi}{4}\PY{p}{,} \PY{l+m+mi}{8}\PY{p}{,} \PY{l+m+mi}{10}\PY{p}{]}\PY{p}{)}
        \PY{n}{s}
\end{Verbatim}


\begin{Verbatim}[commandchars=\\\{\}]
{\color{outcolor}Out[{\color{outcolor}2}]:} 0     0
        1     2
        2     4
        3     8
        4    10
        dtype: int64
\end{Verbatim}
            
    A \texttt{DataFrame} can be created by passing a dictionary.

    \begin{Verbatim}[commandchars=\\\{\}]
{\color{incolor}In [{\color{incolor}3}]:} \PY{n}{df} \PY{o}{=} \PY{n}{pd}\PY{o}{.}\PY{n}{DataFrame}\PY{p}{(}\PY{p}{\PYZob{}}\PY{l+s+s1}{\PYZsq{}}\PY{l+s+s1}{name}\PY{l+s+s1}{\PYZsq{}}\PY{p}{:} \PY{p}{[}\PY{l+s+s1}{\PYZsq{}}\PY{l+s+s1}{John}\PY{l+s+s1}{\PYZsq{}}\PY{p}{,} \PY{l+s+s1}{\PYZsq{}}\PY{l+s+s1}{Ana}\PY{l+s+s1}{\PYZsq{}}\PY{p}{,} \PY{l+s+s1}{\PYZsq{}}\PY{l+s+s1}{Rob}\PY{l+s+s1}{\PYZsq{}}\PY{p}{,} \PY{l+s+s1}{\PYZsq{}}\PY{l+s+s1}{Sue}\PY{l+s+s1}{\PYZsq{}}\PY{p}{]}\PY{p}{,}
                           \PY{l+s+s1}{\PYZsq{}}\PY{l+s+s1}{age}\PY{l+s+s1}{\PYZsq{}}\PY{p}{:} \PY{p}{[}\PY{l+m+mi}{24}\PY{p}{,} \PY{l+m+mi}{21}\PY{p}{,} \PY{l+m+mi}{25}\PY{p}{,} \PY{l+m+mi}{24}\PY{p}{]}\PY{p}{,}
                           \PY{l+s+s1}{\PYZsq{}}\PY{l+s+s1}{grade}\PY{l+s+s1}{\PYZsq{}}\PY{p}{:} \PY{p}{[}\PY{l+m+mf}{85.0}\PY{p}{,} \PY{l+m+mf}{97.0}\PY{p}{,} \PY{l+m+mf}{78.0}\PY{p}{,} \PY{l+m+mf}{90.0}\PY{p}{]}\PY{p}{,}
                           \PY{l+s+s1}{\PYZsq{}}\PY{l+s+s1}{major}\PY{l+s+s1}{\PYZsq{}}\PY{p}{:} \PY{p}{[}\PY{l+s+s1}{\PYZsq{}}\PY{l+s+s1}{Math}\PY{l+s+s1}{\PYZsq{}}\PY{p}{,} \PY{l+s+s1}{\PYZsq{}}\PY{l+s+s1}{CS}\PY{l+s+s1}{\PYZsq{}}\PY{p}{,} \PY{l+s+s1}{\PYZsq{}}\PY{l+s+s1}{CS}\PY{l+s+s1}{\PYZsq{}}\PY{p}{,} \PY{l+s+s1}{\PYZsq{}}\PY{l+s+s1}{ECE}\PY{l+s+s1}{\PYZsq{}}\PY{p}{]}\PY{p}{\PYZcb{}}\PY{p}{)}
        \PY{n}{df}
\end{Verbatim}


\begin{Verbatim}[commandchars=\\\{\}]
{\color{outcolor}Out[{\color{outcolor}3}]:}    age  grade major  name
        0   24   85.0  Math  John
        1   21   97.0    CS   Ana
        2   25   78.0    CS   Rob
        3   24   90.0   ECE   Sue
\end{Verbatim}
            
    The columns of a \texttt{DataFrame} can have different types and can be
displayed using the \texttt{columns} method.

    \begin{Verbatim}[commandchars=\\\{\}]
{\color{incolor}In [{\color{incolor}4}]:} \PY{n}{df}\PY{o}{.}\PY{n}{dtypes}
\end{Verbatim}


\begin{Verbatim}[commandchars=\\\{\}]
{\color{outcolor}Out[{\color{outcolor}4}]:} age        int64
        grade    float64
        major     object
        name      object
        dtype: object
\end{Verbatim}
            
    \begin{Verbatim}[commandchars=\\\{\}]
{\color{incolor}In [{\color{incolor}5}]:} \PY{n}{df}\PY{o}{.}\PY{n}{columns}
\end{Verbatim}


\begin{Verbatim}[commandchars=\\\{\}]
{\color{outcolor}Out[{\color{outcolor}5}]:} Index(['age', 'grade', 'major', 'name'], dtype='object')
\end{Verbatim}
            
    Selecting a single column of a \texttt{DataFrame} yields a
\texttt{Series}.

    \begin{Verbatim}[commandchars=\\\{\}]
{\color{incolor}In [{\color{incolor}6}]:} \PY{n}{df}\PY{p}{[}\PY{l+s+s1}{\PYZsq{}}\PY{l+s+s1}{name}\PY{l+s+s1}{\PYZsq{}}\PY{p}{]}
\end{Verbatim}


\begin{Verbatim}[commandchars=\\\{\}]
{\color{outcolor}Out[{\color{outcolor}6}]:} 0    John
        1     Ana
        2     Rob
        3     Sue
        Name: name, dtype: object
\end{Verbatim}
            
    A subset of rows and columns can also be selected using the
\texttt{iloc} and \texttt{loc} methods.

    \textbf{Task 15 (of 20): Select the first two rows and the last two
columns of \texttt{DataFrame} \texttt{df} using the \texttt{iloc}
method.} \emph{Hint:} The \texttt{iloc} method is used for indexing by
integer position.

    \begin{Verbatim}[commandchars=\\\{\}]
{\color{incolor}In [{\color{incolor}8}]:} \PY{n}{df}\PY{o}{.}\PY{n}{iloc}\PY{p}{[}\PY{p}{:}\PY{l+m+mi}{2}\PY{p}{,}\PY{o}{\PYZhy{}}\PY{l+m+mi}{2}\PY{p}{:}\PY{p}{]}
\end{Verbatim}


\begin{Verbatim}[commandchars=\\\{\}]
{\color{outcolor}Out[{\color{outcolor}8}]:}   major  name
        0  Math  John
        1    CS   Ana
\end{Verbatim}
            
    \textbf{Task 16 (of 20): Select the first two rows and the last two
columns of \texttt{DataFrame} \texttt{df} using the \texttt{loc}
method.} \emph{Hint:} The \texttt{loc} method is used for indexing by
label.

    \begin{Verbatim}[commandchars=\\\{\}]
{\color{incolor}In [{\color{incolor}11}]:} \PY{n}{df}\PY{o}{.}\PY{n}{loc}\PY{p}{[}\PY{l+m+mi}{0}\PY{p}{:}\PY{l+m+mi}{1}\PY{p}{,}\PY{p}{[}\PY{l+s+s1}{\PYZsq{}}\PY{l+s+s1}{major}\PY{l+s+s1}{\PYZsq{}}\PY{p}{,}\PY{l+s+s1}{\PYZsq{}}\PY{l+s+s1}{name}\PY{l+s+s1}{\PYZsq{}}\PY{p}{]}\PY{p}{]}
\end{Verbatim}


\begin{Verbatim}[commandchars=\\\{\}]
{\color{outcolor}Out[{\color{outcolor}11}]:}   major  name
         0  Math  John
         1    CS   Ana
\end{Verbatim}
            
    \subsubsection{Part 8: pandas - Sorting, Grouping, and
Merging}\label{part-8-pandas---sorting-grouping-and-merging}

The values in a \texttt{DataFrame} can be \textbf{sorted} using the
\texttt{sort\_values} method.

    \textbf{Task 17 (of 20): Sort the rows of \texttt{DataFrame} \texttt{df}
by grade from largest to smallest using the \texttt{sort\_values}
method.} \emph{Hint:} Check the parameters of the \texttt{sort\_values}
method.

    \begin{Verbatim}[commandchars=\\\{\}]
{\color{incolor}In [{\color{incolor}14}]:} \PY{n}{df}\PY{o}{.}\PY{n}{sort\PYZus{}values}\PY{p}{(}\PY{n}{by}\PY{o}{=}\PY{l+s+s1}{\PYZsq{}}\PY{l+s+s1}{grade}\PY{l+s+s1}{\PYZsq{}}\PY{p}{,} \PY{n}{ascending}\PY{o}{=}\PY{k+kc}{True}\PY{p}{)}
\end{Verbatim}


\begin{Verbatim}[commandchars=\\\{\}]
{\color{outcolor}Out[{\color{outcolor}14}]:}    age  grade major  name
         2   25   78.0    CS   Rob
         0   24   85.0  Math  John
         3   24   90.0   ECE   Sue
         1   21   97.0    CS   Ana
\end{Verbatim}
            
    The values in a \texttt{DataFrame} can also be \textbf{grouped} based on
some criteria using the \texttt{groupby} method. Then, a function can be
applied to each group independently.

    \textbf{Task 18 (of 20): Group the rows of \texttt{DataFrame}
\texttt{df} by major using the \texttt{groupby} method and find the mean
age and mean grade of each group.}

    \begin{Verbatim}[commandchars=\\\{\}]
{\color{incolor}In [{\color{incolor}22}]:} \PY{n}{df}\PY{o}{.}\PY{n}{groupby}\PY{p}{(}\PY{p}{[}\PY{l+s+s1}{\PYZsq{}}\PY{l+s+s1}{major}\PY{l+s+s1}{\PYZsq{}}\PY{p}{]}\PY{p}{)}\PY{o}{.}\PY{n}{mean}\PY{p}{(}\PY{p}{)}
\end{Verbatim}


\begin{Verbatim}[commandchars=\\\{\}]
{\color{outcolor}Out[{\color{outcolor}22}]:}        age  grade
         major            
         CS      23   87.5
         ECE     24   90.0
         Math    24   85.0
\end{Verbatim}
            
    \texttt{DataFrames} can be \textbf{concatenated} together using the
\texttt{concat()} function.

    \begin{Verbatim}[commandchars=\\\{\}]
{\color{incolor}In [{\color{incolor}23}]:} \PY{n}{df2} \PY{o}{=} \PY{n}{pd}\PY{o}{.}\PY{n}{DataFrame}\PY{p}{(}\PY{p}{\PYZob{}}\PY{l+s+s1}{\PYZsq{}}\PY{l+s+s1}{name}\PY{l+s+s1}{\PYZsq{}}\PY{p}{:} \PY{p}{[}\PY{l+s+s1}{\PYZsq{}}\PY{l+s+s1}{Tom}\PY{l+s+s1}{\PYZsq{}}\PY{p}{]}\PY{p}{,}
                            \PY{l+s+s1}{\PYZsq{}}\PY{l+s+s1}{age}\PY{l+s+s1}{\PYZsq{}}\PY{p}{:} \PY{p}{[}\PY{l+m+mi}{22}\PY{p}{]}\PY{p}{,}
                            \PY{l+s+s1}{\PYZsq{}}\PY{l+s+s1}{grade}\PY{l+s+s1}{\PYZsq{}}\PY{p}{:} \PY{p}{[}\PY{l+m+mf}{88.0}\PY{p}{]}\PY{p}{,}
                            \PY{l+s+s1}{\PYZsq{}}\PY{l+s+s1}{major}\PY{l+s+s1}{\PYZsq{}}\PY{p}{:} \PY{p}{[}\PY{l+s+s1}{\PYZsq{}}\PY{l+s+s1}{Math}\PY{l+s+s1}{\PYZsq{}}\PY{p}{]}\PY{p}{\PYZcb{}}\PY{p}{)}
         \PY{n}{df2}
\end{Verbatim}


\begin{Verbatim}[commandchars=\\\{\}]
{\color{outcolor}Out[{\color{outcolor}23}]:}    age  grade major name
         0   22   88.0  Math  Tom
\end{Verbatim}
            
    \textbf{Task 19 (of 20): Concatenate \texttt{DataFrames} \texttt{df} and
\texttt{df2} using the \texttt{concat()} function.}

    \begin{Verbatim}[commandchars=\\\{\}]
{\color{incolor}In [{\color{incolor}28}]:} \PY{n}{pd}\PY{o}{.}\PY{n}{concat}\PY{p}{(}\PY{p}{[}\PY{n}{df}\PY{p}{,}\PY{n}{df2}\PY{p}{]}\PY{p}{)}\PY{o}{.}\PY{n}{reset\PYZus{}index}\PY{p}{(}\PY{p}{)}
\end{Verbatim}


\begin{Verbatim}[commandchars=\\\{\}]
{\color{outcolor}Out[{\color{outcolor}28}]:}    index  age  grade major  name
         0      0   24   85.0  Math  John
         1      1   21   97.0    CS   Ana
         2      2   25   78.0    CS   Rob
         3      3   24   90.0   ECE   Sue
         4      0   22   88.0  Math   Tom
\end{Verbatim}
            
    Alternatively, rows can be added to a \texttt{DataFrame} using the
\texttt{append} method.

\textbf{Task 20 (of 20): Add \texttt{DataFrame} \texttt{df2} to
\texttt{DataFrame} \texttt{df} using the \texttt{append()} method.}

    \begin{Verbatim}[commandchars=\\\{\}]
{\color{incolor}In [{\color{incolor}30}]:} \PY{n}{df}\PY{o}{.}\PY{n}{append}\PY{p}{(}\PY{n}{df2}\PY{p}{)}\PY{o}{.}\PY{n}{reset\PYZus{}index}\PY{p}{(}\PY{p}{)}
\end{Verbatim}


\begin{Verbatim}[commandchars=\\\{\}]
{\color{outcolor}Out[{\color{outcolor}30}]:}    index  age  grade major  name
         0      0   24   85.0  Math  John
         1      1   21   97.0    CS   Ana
         2      2   25   78.0    CS   Rob
         3      3   24   90.0   ECE   Sue
         4      0   22   88.0  Math   Tom
\end{Verbatim}
            

    % Add a bibliography block to the postdoc
    
    
    
    \end{document}
